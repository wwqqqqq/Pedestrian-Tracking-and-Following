% !TeX root = ../main.tex

\chapter{以可佳机器人为基础的行人追踪系统}

\section{输入设备}

\subsection{Kinect}

Kinect深度相机,提供RGB图像和深度图像。利用红外线成像,受自然光照影响较大。

\subsection{2D激光}

用于定位和导航

\section{视觉追踪系统}

\subsection{总体架构}

\subsection{目标人物注册}
1. 使用OpenPose系统找到目标人物

给人物设定一个初始的手势,由OpenPose系统识别出人物的骨架,进而提取出视野中所有人物的姿势。如要求目标人物将一只手举起,直到机器人识别出目标人物,判断出追踪区域(Region of Interest, ROI),并提示“识别成功”,即开始追踪。

2. 建立一个判别器:根据已有的ROI作为正样本,随机提取背景作为负样本,通过在线学习拟合出一个分类器。该判别器的作用是当追踪失去目标或出错时,当再在图像中检测到一个新人物时,判断对方是否是一开始的目标人物。

\subsection{行人追踪}

1. 使用KFC和CSRT,根据“目标人物注册”中得到的ROI进行目标追踪。

\subsection{目标丢失恢复}

1. 全局扫描找到行人,结合“目标人物注册”中的判别器,判断是否是目标人物。

2. 结合目标人物最后出现的位置,类似TLD?

