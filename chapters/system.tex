% !TeX root = ../main.tex

\chapter{以可佳机器人为基础的行人追踪系统}

\section{输入设备}

\subsection{Kinect}

  Kinect深度相机可以同时提供RGB图像和深度图像。

  使用RGB-D相机Kinect作为彩色图像和深度信息的输入,而Kinect的深度图像只能有效判断80厘米之外的物体的距离,且由于Kinect通过红外发射器和红外摄像头来获得深度信息,所以受阳光照射的影响较大,所以在实际使用中,Kinect深度图像可能会出现不稳定和空洞现象。

对于Kinect的深度图像在近距离精度较低的现象,考虑在较近距离内使用2D激光进行行人的3D位置判别。对于Kinect的空洞现象,考虑使用KinectFusion算法\cite{newcombe2011kinectfusion}或OctoMap\cite{hornung2013octomap},将深度图像投影到RGB图像中,以进行对相机视野中场景的3D重建。

\subsection{2D激光}

  由于本文中的行人检测与追踪算法主要使用了视觉信息,所以这里2D激光主要用于建图、定位和导航。定位和导航在本文中不是主要内容,所以不再特别赘述。



\section{视觉追踪系统}
\subsection{总体架构}

\subsubsection{目标人物注册}
\paragraph{使用OpenPose系统找到目标人物}

  追踪系统在初始化时需要得到目标行人所在的大致位置,即提供行人在初始帧中的界限框(bounding box),并由此对追踪器和分类器进行初始化。为了能够有效地对目标用户进行识别,这里采用开源系统OpenPose\cite{cao2018openpose}来进行人体的骨骼识别。OpenPose系统不仅可以判断出图像中所有人物的骨骼信息,还可以对人物的姿势进行识别,这带来了另一个好处,即我们可以要求被跟随的目标站在机器人前方做出一个指定的姿势,如举起右手,直到机器人提示注册成功,这样一来,我们就可以在对目标人物的特征没有先验知识的情况下在初始化时发现目标了。

\paragraph{初始化分类器}
  根据第一帧由 OpenPose 识别出的目标人物所在区域作为正样本,随机提取背景作为负样本,通过拟合出一个分类器。该判别器的作用是在追踪算法失败后负责进行全局搜索和恢复。

  根据之前所调研的结果,这里采用 HSV 直方图和 HOG 特征结合,作为图像的特征向量和分类器的输入。由于对人物的先验知识较少,且人物可能在追踪过程中发生姿态变化等影响检测算法准确度的形变,采用 Online AdaBoost 算法,以行人在每帧的外观作为样本不断进行学习,更新分类器。

\subsubsection{行人追踪}

\paragraph{行人追踪算法的比较与选择}

  只测试了KCF、CSR-DCF、TLD、MedianFLow和MOSSE算法。在用于行人追踪时,在帧率上,MOSSE算法以约450FPS的帧率遥遥领先,其次是 270FPS的MedianFlow,40FPS的KCF算法,TLD和CSR-DCF算法的帧率较低,在15到20FPS之间,但也勉强可以达到实时水平。在追踪准确度上,KCF算法在尺度不变时准确度较高,能在一定程度上解决目标被遮挡的问题,且有可能在失败后再次恢复,但前面已经讨论过,KCF算法无法适应目标尺度大小的变化;TLD算法识别准确率较低,边界框会经常跳动或漂移,甚至会经常检测到错误的位置,但在目标在视野中消失一段时间后再回到画面时,只有TLD算法能识别到目标的回归并重新建立起追踪;CSR-DCF算法识别准确度较高,能够稳定地检测目标的位置,且与KCF算法相比尺度可变,但当检测失败后,CSR-DCF算法几乎无法恢复;MOSSE算法同样是尺度不变的追踪,对于遮挡的情况不鲁棒,追踪不准确,且对于错误的追踪无法识别和恢复;MedianFlow算法准确度相对也不高,在失败后无法恢复。

  综合这些追踪器的表现,主要考虑使用CSR-DCF算法,但同时,针对它难以在失败后恢复的缺点,需要额外判断它在追踪时出现的错误和失败后的恢复。

\paragraph{更新分类器}

  在每一帧都会由追踪器得到目标所在的图像块,由分类器给出一个它符合行人特征的得分,当得分超过阈值$threshold_1$ 时,认为它是完全准确的追踪,并将其作为正样本加入分类器进行训练;当得分低于阈值$threshold_2$时,认为这个29中国科学技术大学本科毕业论文追踪是失败的,使用分类器对图像进行扫描来搜索目标,搜索到,即用新的边界框重新开始追踪,若没有搜索到,即认为目标搜索失败。

\subsubsection{目标丢失恢复}

  追踪失败有几种可能:(1)目标从视野的左右两侧消失;(2)目标被完全部分或完全遮挡;(3)目标进入另外的房间。

  对于可能(1)根据目标消失的位置及时转动机器人的视角,如果追踪器还是没能恢复追踪,则使用分类器在每一帧上对目标进行搜索,直到找到目标人物。对于可能性(2)(3),则使机器人尽快前往目标出现的上一个位置,并将视野转向行人可能所在的地点,用分类器进行目标搜索。