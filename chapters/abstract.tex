% !TeX root = ../main.tex

\begin{abstract}
  服务机器人是一种半自助或全自主工作的机器人。它能完成有益于人类健康的服务工作,但不包括从事生产的设备。服务机器人分为个人/家庭服务机器人和专业机器人。专业机器人一般在特定场景中使用,如商业服务、物流、医疗、救援等;而个人/家用服务机器人主要在日常生活场景中进行与人进行交互,提供家政服务、陪伴、娱乐、辅助学习等多种功能,包括家政机器人、娱乐休闲机器人、助老助残机器人等。其中,个人/家庭服务机器人为本文研究内容所适用的对象。

  为了精准理解当前环境和有效执行指令,能够精确可靠地自动识别目标人物并对其进行追踪陪同,是移动服务机器人的人机交互中的一项重要且必要的功能。

  本文将针对室内移动机器人的行人跟随问题做如下研究:

  (1)常用目标跟随算法的原理与实现;

  (2)ROS导航和可佳导航介绍;

  (3)可佳机器人上行人跟随系统的实现。

  \keywords{计算机视觉;机器人;目标追踪;路径规划}
\end{abstract}

\begin{enabstract}
  A service robot is a robot which operates semi- or fully autonomously to perform services useful to the well-being of humans and equipment, they exclude manufacturing operations, and they are capable of making decisions and acting autonomously in real and unpredictable environments to accomplish determined tasks.  There are two types of service robots, personal/domestic service robots and professional robots. Professional robots are typically used in specific occasions, including business, delivery, medical, rescue, etc. Personal/domestic service robots, which include cleaning robots, elder care and medical companions, entertainment and leisure robots, home education and training robots, are the specific research objects in this paper.

  In order to accurately understand the current environment and effectively execute the instructions, one of the important and necessary ability for a personal service robot, is to automatically recognize and track a person precisely and robustly.

  This paper is consist of three parts:

  (1) The theories and implementation of various existing object tracking algorithms;

  (2) Introduction of ROS navigation and kejia navigation;

  (3) The implementation of the people following system on kejia robot.

  \enkeywords{Computer Vision; Robotics; Object Tracking; Path Planning}
\end{enabstract}
