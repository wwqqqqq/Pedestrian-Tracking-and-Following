% !TeX root = ../main.tex

\chapter{绪论}

\section{研究背景及意义}

  服务机器人是一种半自助或全自主工作的机器人。它能完成有益于人类健康的服务工作,但不包括从事生产的设备。它们可以在真实且不可预测的环境中自动进行决策和行动来完成确定的任务。服务机器人分为个人/家庭服务机器人和专业机器人。专业机器人一般在特定场景中使用,如商业服务、物流、医疗、救援等;而个人/家用服务机器人主要在日常生活场景中进行与人进行交互,提供家政服务、陪伴、娱乐、辅助学习等多种功能,包括家政机器人、娱乐休闲机器人、助老助残机器人等。其中,个人/家庭服务机器人为本文研究内容的应用对象。

  随着人工智能与物联网技术不断发展,服务机器人作为智能硬件之一,不断地丰富其自身功能及其实现更强大的性能,服务机器人市场也在迅速发展,家庭用机器人、智能公共服务机器人的应用场景和服务模式不断拓展。为了精准理解当前环境和有效执行指令,能够精确可靠地自动识别目标人物并对其进行追踪陪同,是这类移动服务机器人的人机交互中的一项重要且必要的功能。这项功能可以用于在博物馆或医院等场景中对用户进行指引,在家庭中陪伴用户、与用户进行交互,助老、助残等。

  移动机器人的核心技术包括导航定位、地图创建、路径规划、任务分配和目标跟踪等。在本文中,会主要介绍目标跟踪,尤其是其中的行人定位模块,通过对ROS导航的简要介绍涉及机器人导航定位、地图创建、路径规划的大致概念,并最终实现一个有简单导航模块和较为鲁棒的行人追踪模块的机器人系统。

\section{相关工作}

  自从有了服务机器人的概念以来,就有了很多对于机器人目标跟随技术的研究。移动机器人由于可携带多种传感器,由此衍生了各种使用了传感器的行人识别和追踪方法。如使用视觉图像、激光传感器、热成像传感器\cite{treptow2006real}、声音传感器\cite{zhou2008target}等,由于单一传感器信息较少,也出现了不少多传感器融合的行人定位方式\cite{susperregi2013rgb}。

  激光传感器由于还被广泛运用到机器人的定位、建图、导航中,是机器人最为重要的传感器,所以也最多地被运用到行人跟随的任务中。不过激光传感器由于成本限制,多使用2D激光检测同一水平面的物体,竖直空间上可以检测到的范围有限,所以激光行人追踪的方法最为经常使用的特征是行人的双腿\cite{arras2008efficient}。在激光图像中,人的双腿在有一种明显的模式,通过滤波器抽取人的腿部特征,进行聚类,便可得到一个人腿模式识别器,再结合粒子滤波或卡尔曼滤波等技术,便可以实现对行人腿部的持续追踪。但这种识别器在有遮挡时,或暂时失去目标后,无法准确地重新建立对目标的追踪。此外,从人腿信息中基本不可能对不同行人加以区分,所以为了系统的鲁棒性,行人腿部识别通常还是需要加入视觉信息,如结合面部识别,来进行一些情况下的行人的区分\cite{kleinehagenbrock2002person,bellotto2008multisensor}。

  视觉行人追踪中也有很多已有的方法,包括对行人的面部识别\cite{wong1995mobile,cruz2008rea}(要求行人必须面对相机),衣着识别\cite{bellotto2008multimodal,noceti2009combined}(主要通过对衣着颜色特征的提取),身体轮廓识别\cite{}(使用HOG、边缘描述子等特征),行人步态识别\cite{mowbray2003automatic,koide2016identification}等。

  但在现有的大多行人追踪算法中也存在一些问题,如在行人在视野中消失或被遮挡一段时间后,追踪器如何从失败中恢复。本文在接下来会对行人检测和追踪常用的特征进行分析,选择合适的特征和追踪器实现一个视觉行人追踪系统,并尝试解决追踪器丢失目标后的恢复问题。


