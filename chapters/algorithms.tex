% !TeX root = ../main.tex

\chapter{行人追踪算法}
  行人追踪算法blabla

  
\section{视觉行人追踪}
  计算机视觉中的行人追踪,主要包括密集跟踪方法,即基于行人检测和识别的追踪,以及稀疏跟踪方法,即基于目标动态的追踪。

  在密集跟踪方法中,我们实际上并没有“跟踪”物体,而是在视频不同的时间点的一系列帧上扫描和检测物体的位置。由于每次的目标检测都是独立地在当前帧上进行的,所以每次检测时,都需要处理图像中的所有像素,所以以这种方法进行目标跟踪,计算量会比较大。

  稀疏跟踪方法是根据物体的动态信息,对其可能的运动轨迹进行预测,并结合其上一帧所在位置和对当前帧的观察,得出其当前位置的算法。由于已知物体在上一帧时的位置,所以对当前帧识别时,只需要检测上一帧物体所在位置附近的像素,这样一来,相对于密集跟踪方法,就减少了大量的计算。此外,由于我们结合了对物体运动的预测和观察来进行估计,在一些情况下准确度也会较高,但在物体速度较快时,可能会失去对物体的追踪,当目标物暂时从视野中消失时,可能难以重新找回物体。

\subsection{基于检测的追踪}

\subsection{基于动态的追踪}


\section{激光行人追踪}


