% !TeX root = ../main.tex

\chapter{总结}

  通过对于已有机器人行人跟随解决方案的调研,本文确定了从视觉行人追踪算法为基础的机器人行人跟随方法。在视觉行人追踪领域,调研了常用的包括卡尔曼滤波、粒子滤波、均值漂移、相关滤波等视觉目标追踪算法,经过实验后,最终选定了相关滤波中的CSR-DCF算法作为本文系统中追踪器的实现算法。CSR-DCF算法对于遮挡、形变、光照都在一定程度上鲁棒,且可以适应目标的尺度变化,是目前视觉追踪中准确度和速度都较为优秀的算法之一。但由于CSR-DCF算法存在对目标失去追踪后难以恢复的问题,且在复杂情况下可能会出现错误的追踪,本文还调研了一系列单帧上的行人检测算法,并在系统中加入使用HOG结合HSV直方图作为特征,SVM作为分类器的行人检测器,以防止追踪器发生误判或漏判现象,以及帮助追踪器从追踪失败中进行恢复,保证系统能在多数情况下正确运行,且能够保持长期的追踪。

  本文最终实现的视觉行人追踪系统比起仅使用CSR-DCF算法追踪的系统,可以在一定程度上处理追踪器错误追踪其他行人或物体的情况,并且当追踪器失去对目标的追踪后,只要目标回到画面中,便可以很快地恢复追踪。同时,运行速度在大部分情况下可以达到实时水平。

  此外,本文还调研了ROS中的导航系统,包括ROS提供的开源导航包和由中国科大研发的可佳导航系统。在使用视觉行人追踪系统得到行人在RGB图像中的位置后,通过对齐Kinect深度图像得到行人在机器人坐标系中的三维位置,设置为目标后再调用导航系统中的导航功能便可以实现对于目标跟随。